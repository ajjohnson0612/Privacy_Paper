\documentclass{article}
\usepackage[utf8]{inputenc}
\usepackage{graphicx}

\author{Austin Johnson \\
\texttt{ajohns20@uccs.edu}
\and 
Thomas Mcdonald \\
\texttt{tmcdonal@uccs.edu}
\and 
Hunt \\
\texttt{email@uccs.edu}
}

\title{Targeted Ads from Search Engines}
\date{April 2019}


\usepackage{multicol}


\setlength{\parskip}{1em}

\begin{document}
\maketitle


\section{Abstract}
\small
\textit{
With many different Search engines available to browse the modern day internet, we show the difference in tracking done from the search engines and how they show targeted advertisement based on what was searched for by the user. By typing in different key words to a search engine and recording whether the keywords showed up in advertisements while tracking after how many searches the ads appeared. We found that the search engine DuckDuckGo had the least amount of advertisement shown for the keywords entered. }    

\section{Introduction}
\small
\textit{
	Search engines have always been a huge part of the web. The computer revolution would've never happened if it was'nt for yahoos inovative way of scraping the web for relevant information. Since those days, however, the web has become more and more personalized depending on the individual. Our search engine results and advertisements are a prime example of the internet tracking and controlling our user experience. The outrageous amount of data tracking spanning accross companies gave our team the idea to see if there is a correlation between search queries and targeted advertisements. Our initial hunch is google search communicates with google analytics and can cater advertisements relevant to what the user has frequently searched. To further our study, we are also going to see if other search engines share information to advertisement software as well.  
}

\section{Design}
For the design of our experiment, we decided scripting our search results was the fastest and best way to get targeted advertisements. To conduct the experiment, we set up a virtual machine with a fresh install of the latest stable release of Mozilla FireFox to remove all possibility of previous searches potentially influencing the shown Advertisement. For the experiment, we created a list of eleven different key words that were are to be used as search parameters in a specific search engine. Our script then opens 50 new tabs opened with the search results of each key word. To conclude, we use a list of common websites that are known to be heavily populated with advertisements and see how many targeted advertisements are correlated with our past search queries. 

\begin{tabular}{ c c c }
	
	\hline
	
	Tide Pods     & Mercedes Benz            & Apple Watch \\ 
	Dell Computer & Bose Headphones          & Stride Gum  \\  
	Colt AR15     & American Spirits         & Sunglasses  \\
	Condoms       & Colorado Springs Roofing &             \\
	
	\hline
\end{tabular}



\textbf{Google}

\textbf{Bing}

\textbf{DuckDuckGo}

\section{Method}
\small
\textit{
	Each team member of our group used our template script changed the url to the search engine we were testing. See the figure below (screencap of code in mintos). After running the script, we went to some websites that are known to have alot of targeted advertisements. The list of websites is:
	\begin{itemize}
		\item https://www.theverge.com/
		\item https://www.foxnews.com/
		\item https://www.gamespot.com/
		\item https://stackoverflow.com/
		\item https://www.reddit.com/
		\item https://www.youtube.com/
		\item https://imgur.com/
		\item https://www.amazon.com/
		\item https://www.buzzfeed.com/
		\item https://www.ign.com/
	\end{itemize}
	We visited each site on our virtual box and tallied which websites had a relevant advertisement to our search results. The results for each search engine differ.
}

\section{Discussion}


\section{Conclusion}

\end{document}